\documentclass{article}

\title{CS 452 Project 2 Documentation}
\author{
  Avi Itskovich, 20332164
  \and
  Alex Ianus, 20342535
}

\begin{document}

\maketitle

\section{Overview}

This documentation covers milestone two of our train project.

\section{Operating Instructions}
\begin{enumerate}
  \item Reset the box and wait for the redboot prompt
  \item load -b 0x00218000 -h 10.15.167.4 \$executable
  \item go
\end{enumerate}

\section{Submitted Files}
Root directory: /u1/aianus/cs452/handin/project2/

\subsection{Executable}
\begin{verbatim}
0d572c6fe87cf82fc50e29045c3f7214  project2.elf
\end{verbatim}

\subsection{Code}
\begin{verbatim}
d39148360f7d205cb98a36d0a0c61aad  ./include/bwio.h
339437283fc2f0d78f47a74f4fe61a52  ./include/circular_queue.h
9f6106bd19009aec00d7f458e964d719  ./include/ts7200.h
3d308fffb8d7c9cdaba180f4867bb516  ./include/ksyscalls.h
081b58b1ad6e401d9648cb88ffb30301  ./include/request.h
ccac3c439d86709412554719f50434cc  ./include/switch.h
9aef3c967205b2c838b9ce1dbe0704a3  ./include/task.h
7cc98e8bfe2af9dfa5426aeb8da10e92  ./include/nameserver.h
18060a9170044e77e2c91364f48054f8  ./include/scheduling.h
01b23f49792d910a524f42badf460c5c  ./include/libc/syscall.h
01c2b8c8738d48493c9e57e216e86ff0  ./include/libc/dassert.h
5c61838a4125651ebe41601ccd881455  ./include/libc/random.h
e31e1aa5f0cf7747c5ca02d9219265d0  ./include/libc/fine_timer.h
019ef4d963e27b73dc863cd252201dce  ./include/libc/heap_priority_queue.h
4e13693535c5e92f1d0041b56ce2d557  ./include/libc/log.h
b9124cb9c114764f4ac6fb6ae00f8d1d  ./include/libc/ring_buffer.h
2fa1c69b76a787a5b5dc6b480aa5d734  ./include/libc/sprintf.h
d1418a29bf14f9cffdc17b4b486256f3  ./include/libc/conv.h
1dee5d1af1534d6eb63e1c60a39482b1  ./include/libc/string.h
ac4459311ef2e5e5d607c556a0db393d  ./include/libc/encoding.h
7b9a8374db9a2a3211ef11a5ac00f20e  ./include/libc/memcpy.h
78ac3cf1f6b60961ca67e33dfc701c64  ./include/libc/memory.h
4832f57aab043ea00409f6cb2025e3cd  ./include/libc/track.h
be1297d4e484ad68e9581359902968c8  ./include/libc/common.h
0bb9501743135b42ee43e46949f6cd2f  ./include/libc/stack.h
e4afa944926d4fab745f25de33535d64  ./include/nameservice.h
af46e41f00d29010abdca05eafe75609  ./include/verify.h
f5d65e70556962f5effba160e9bbd0a0  ./include/messaging.h
7a43181f7908eef779e270832fcd9148  ./include/bits.h
a53c63ad1757b25362486bce6ef7c90d  ./include/user/rps/rps.h
570c77afbd19a6a0be4419d136ee9a1f  ./include/user/rps/rpsclient.h
331b47449d891d458b0647da2c2924de  ./include/user/rps/rpsserver.h
862df1e336ef89d83bceadc522085dd1  ./include/user/user.h
92023eb7f5b51008caee8b80306fde73  ./include/user/clock/clock.h
67d3408c795e016994abc7daccaf2dfb  ./include/user/clock/clock_notifier.h
e9f7e445c32353f3b72917ecf553de05  ./include/user/clock/clock_server.h
e75d3afab221f201593235fa63ae912e  ./include/user/clock/clock_widget.h
c43444432320abfb09777821a38ae8a7  ./include/user/idle.h
6320a395d0add67f42fb141fb8301591  ./include/user/serial/read_notifier.h
e3c465f3ae1a65a8bf6e144372e9f2b4  ./include/user/serial/read_server.h
3e6ecd73503f1abcbf7e33a574780d95  ./include/user/serial/read_service.h
2d8b7da4e01baff3d765879657ae984b  ./include/user/serial/serial.h
1481dd9fe1dcd2ad9f5328171ebec4b9  ./include/user/serial/write_notifier.h
4b8c17d07abc40b0e36874eb758663e7  ./include/user/serial/write_server.h
3ab9dd46e2c0b4d73b4fb2d38bb72006  ./include/user/serial/write_service.h
994b625e8810a90d433639283dee842b  ./include/user/calibration/calibration.h
f6fbf657c61b70cedaa8a43f3c339536  ./include/user/calibration/velocity_calibrator.h
33329917168536461949d2020b837b12  ./include/user/shell.h
d2ffb3543bad05bcf695c4cde2aa738f  ./include/user/switch_server.h
28dfb869aef2191e1dc6e4bcc1cd144e  ./include/user/distance/distance_notifier.h
c688d7be0a097a4f7247394b2d532172  ./include/user/location/location_service.h
a3d9a640af567a20a59695a18760283d  ./include/user/location/location_server.h
79a75d577e35fa3c4932b55aecab21da  ./include/user/mission_control.h
2d7038f2cf03750c88de76d0546d7673  ./include/user/sensor/sensor_service.h
dd0af4b68ec86e59475b5db33fda4a46  ./include/user/sensor/sensor_notifier.h
4b0de334f1b4f3315f89283ce0647a6e  ./include/user/sensor/sensor_server.h
3c0e19f904f5e271f542704790630ba4  ./include/user/sensor/sensor_widget.h
d805d625d50dce647fefd8ad5d315b36  ./include/user/train_task.h
c5f4c7bc8b1e886e0ef71fa164df8dd0  ./include/user/train_widget.h
aee19e03d88ba8a5315e04d7541707e5  ./include/user/pubsub/courier.h
dd7c6bbf2bf50dbb5c730754d2f331f5  ./include/user/pubsub/pubsub.h
a86f4b3b9192c3a531aaeea86304e709  ./include/user/reservation_server.h
ad54c11df98adfdb38617ec59a7f8a7a  ./include/constants.h
5bd4a0357d0702750838dc1417cd4f14  ./include/event.h
cdf8d943e1d25a7d7c9fc693a45256e2  ./include/track/track_node.h
95ff6f43aac027e9832905d4afdaa8c5  ./include/interrupt.h
94833a9ccce763038303123e95428de8  ./include/waiting.h
3e9800bc664c0412dc73529ab1d1cc42  ./include/uart.h
7df60a6a44bebe110c9b245dec42eedc  ./src/bwio.c
ae7ddbbf11a46e10437e256ff3fbdbb5  ./src/kernel.c
c5a60a7707f13874ee5672d221ec2bac  ./src/circular_queue.c
3d07be1b7e34d5dddb0f2981f80eeed9  ./src/switch.c
5ad86d57a8815700222af7812f181477  ./src/task.c
1a456482cd1adc5d5880d60ab9cfdcf3  ./src/ksyscalls.c
a21461fa9ad938328c6399384d82693c  ./src/nameserver.c
3534459591a48a06217bf141d069f3a9  ./src/nameservice.c
1d22f65e42f1c0e9a3fd0b7f7c53dbb7  ./src/libc/memory.c
996a06e15c77de128da322b4e03ec471  ./src/libc/syscall.c
df0e33ec5df1306650e909e7463c3834  ./src/libc/fine_timer.c
428d29b72bb432d11e63812e3f19dedd  ./src/libc/random.c
161c3e1bda2313818a7deb4ab1dfa18e  ./src/libc/heap_priority_queue.c
4a47407f1350444a938b4abdc567e2a9  ./src/libc/log_arm.c
686d7bdcad6b2a66f66e983565ffc8df  ./src/libc/log_x86.c
a5c16093637f6f52699641447bdb536f  ./src/libc/ring_buffer.c
e10d105c3f5b44d0d94e23c278304952  ./src/libc/swi_arm.c
b7bebaa79148664424ff9f0f55c2415c  ./src/libc/swi_x86.c
cf3630c1a255e94e6c4a2979273d25dd  ./src/libc/conv.c
64a7c8eb92ea1295062e2b4b41d32a8f  ./src/libc/sprintf.c
7964dd8d1e3b8eb16c2dbd01e8be0dc4  ./src/libc/string.c
e4e1d07e4d86c82d62f26d82129c70a8  ./src/libc/memcpy.c
2b1855aea8c00d54976391b8c56d8de5  ./src/libc/track.c
9b38b5c16bb0c3c6ef2f2da3024c5d23  ./src/libc/stack.c
cbc4b3cdae62ea6a211c5a0f834e646c  ./src/scheduling.c
b0c8b9d56c401dc19fa710cf4b0b3a53  ./src/test/circular_queue_tests.c
ce32d193c68b271e8167099dd4c32835  ./src/test/ksyscalls_tests.c
0b5b1b5e8febfbd437bb0b775640cb0a  ./src/test/memory_tests.c
e8617b00b01d3540ae62a140e4b2c15c  ./src/test/messaging_tests.c
be5d61656c6f152bd9ecd52c3dec4614  ./src/test/scheduling_tests.c
7fba2ea44d6922b270aff6e5423683e4  ./src/test/priority_queue_tests.c
5601613cdbacddf01dc51029acb20515  ./src/test/unit.c
7503c94508e4383df74166701f79bd4d  ./src/test/pathfinding_tests.c
1e3f84b77a97255d6251ddf262835725  ./src/user/kernel1.c
b23bda2d5f268ed1a85959e266882360  ./src/user/kernel2.c
46a424731361d6176718c8b7c737bc55  ./src/user/rps/rpsclient.c
e8a44b7c42f0a0d16eaf0594e1a02bd3  ./src/user/rps/rpsserver.c
fa6b70558103fa109c508faa51d5071e  ./src/user/timings.c
95197606aa508ee5043c16021c0f6234  ./src/user/user.c
d05ccb77f412ed3b343094900c142f04  ./src/user/idle.c
d8e09c1a0766a1ee4d3f8c6b9c8dc7f8  ./src/user/kernel3.c
5a69a467671b083ba03266175cad6090  ./src/user/clock/clock_notifier.c
1441403383f40c229ef151ed13b68993  ./src/user/clock/clock_server.c
df32193e4f6a93c79d7f647d40f5cdee  ./src/user/clock/clock_widget.c
f7cd27ec82053c8edd80b7f92b214f05  ./src/user/kernel4.c
685878716ba1524a54f952330e6caac3  ./src/user/serial/read_notifier.c
327b0c47ddea28ca88b82ef5e3eb6eee  ./src/user/serial/read_server.c
6a63592be4452ed022f8e8cdcaf51391  ./src/user/serial/read_service.c
ceb6691f814c0348cf4b0f9b9699e88d  ./src/user/serial/serial.c
cd47b895551dcb88cd74e0fad3d1584b  ./src/user/serial/write_notifier.c
eb2bf4d21085abca9c52a6d8a9b1d993  ./src/user/serial/write_server.c
11dda1ee3994b49e83aab3dea1498f6b  ./src/user/serial/write_service.c
47adc1b6e70cb59265e87bdab807b50f  ./src/user/calibration/calibration.c
ffc033605d65cc62325b69138804f685  ./src/user/calibration/velocity_calibrator.c
d92e7f78b9756c5d1bf989cff15b0347  ./src/user/reservation_server.c
9b075bebfb12a28fe3a492018e49e061  ./src/user/switch_server.c
bebb7808f7e5160d9572f9fc7ef1a0ea  ./src/user/distance/distance_notifier.c
5524b3ef80bc5d7dd7bf46ccd10ad913  ./src/user/location/location_server.c
f63ef075c59babd964d1319fada23bbe  ./src/user/location/location_service.c
2fc7df9b48ce3ed6f2c52cc3ffa31c2e  ./src/user/mission_control.c
5959f5754d4f908a467ed39c44b3a414  ./src/user/project1.c
04849c38bbc8e2ee9c4559a9d16b463f  ./src/user/sensor/sensor_server.c
12e360f0ae4dd3fcee8a85bdb83ea201  ./src/user/sensor/sensor_notifier.c
7c8905da64b641b0046e9d92522de072  ./src/user/sensor/sensor_service.c
5092ed24bec172bebfdeb60e27e6ef3a  ./src/user/sensor/sensor_widget.c
5286807120a982880bb09a03690c38ab  ./src/user/shell.c
435bdd3b01b1714a26d1f9bf50b1ba9c  ./src/user/train_widget.c
abfadf2d6f31459c4ff9fe0e6cf3ecbd  ./src/user/pubsub/courier.c
73e9e31fcee0bdf641825ba7a705db0f  ./src/user/pubsub/pubsub.c
8dd31e002a0737429634f173570db202  ./src/user/train_task.c
77697972cdac93a02951f8476ce29379  ./src/messaging.c
c946f75ca4551a445ac3eda46091b7be  ./src/bits.c
5337762b0342e2aa9406385f23991cfe  ./src/event.c
0908fa9125745315f2d4bc258242aecc  ./src/interrupt.c
f90b3c33dead7ca55d9192d8112cdc4f  ./src/waiting.c
75dd26305bdbdd505590ec82cb80c6d7  ./src/verify.c
c0554c00f8b3d0589f461dd9323524f7  ./src/uart.c
\end{verbatim}

\section{User Program Description}

\subsection{PubSub System}

Since we use a publish-subscribe model in our system, we made a generalizable system to allow any server to support this. Previously, each server implemented PubSub by itself, and only had one courier for subscriptions. Our new system has 3 different couriers which at different priorities. These allow a low priority task like the train\_widget to subscribe to the same stream as the train controller without a performance loss. Our pubsub sustem supports the following API:

\begin{itemize}
  \item int CreateStream(char *name) - Creates the set of tasks and returns the tid of the PubSub server it created. The name is assigned to the PubSub server in the NameServer.
  \item void Subscribe(char *name, int priority) - Subscribe to a stream. You will begin recieving updates after calling this.
  \item void Unsubscribe(char *name, int priority) - Unsubscribe from a stream. You will recieve no more updates after calling this,
  \item void Publish(tid\_t stream, Message msg) - Publish a message to a stream. This message gets forwarded to all subscribers of the stream. Ordering of messages is guaranteed as fifo.
\end{itemize}

This lets us clearly separate different components of the system and allow them to communicate through streams. It also allows us to easily introspect these streams so we can show the data on the screen.

\subsection{Calibration}

For this milestone we re-calibrated our trains. We created a calibration program for calibrating the average speed of the train. This program calibrates speeds from 2 - 14, by averaging speeds measured over multiple sensor hits. We also generated new acceleration models for the trains. These were generated using the assumption of quadratic acceleration. We measured how long it took the train to accelerate in ticks and using that along with it's velocity at that speed to generate the quadratic acceleration. We also did some manual tweaking of this, waiting a couple of ticks before using this function to better map the train.

\subsection{Sensor Attribution}

Sensor attribution now supports multiple trains. Each train maintains a list of pending sensors that it is expecting to hit. A hit of a sensor that is not associated with any train is considered a spurious hit. Two trains may be waiting for the same sensor. We solve this conflict by giving the sensor to the train with the highest velocity. This isn't a good solution. However, it fixes the issue of having the trains start up at the same siding. A better solution would be to use the train's distance along an edge for this. We just haven't gotten around to implementing it yet.

Internally we augment this distance along an edge every tick based on a trains current velocity. Once we have reached the end of an edge we switch to the next one. If the source of next edge is a sensor, we try to wait for a sensor to fire before switching to that edge. We will wait up to 10cm of extra distance before switching to that edge. If we do switch to that edge, we will record a missed sensor. During sensor attribution this is also checked before a sensor hit is deemed spurious. If we are going to miss a second sensor, we do not switch edges again. This is necessary, because otherwise a stall can cause us to think we're just missing all sensors in a path instead of actually being stalled.

\subsection{Switch Failure}

Switch failure is handled by having each train have as it's pending sensors all sensors down both sides of the branch. This way if we trip either of them we will know where we are. This seems to work quite well in practice. However, it's useless in places where one branch has no sensors.

\subsection{Shell Augmentation}

We've added some simple augmentations to our shell system:
\begin{itemize}
  \item The go command now accepts multiple train/destination pairs.
  \item We've added an orient command so we can inform the system of the location of the pickup relative to the train's direction of travel
  \item We've added a calibrate command to facilitate train calibration.
\end{itemize}

\subsection{Train Widget}

During working on this demo we realized our train widget was printing much too often. We instead moving to a differential model, where we only print what's changed. However, this still wasn't enough becuase things were changing too quickly. So we further expanded on that to have it only print what changed every 10 ticks

\subsection{Reservation System}

We have a separate reservation task in order to make reservation operations atomic. The operations supported are Reserve, Release, and SwapForReverse. The Reserve operation is used to reserve a section of track. We represent sections of track by their source nodes. As part of reserving a section of track the opposite direction is also implicitly reserved. Reservations are re-entrant, so a train is allowed to reserve track it already owns. If someone else owns the track, a failure is immediately returned.

Release is used to give up a section of track. Implicitly reserved nodes are also released. An error is returned if the caller does not in fact own the track it is trying to release.

SwapForReverse is used to atomically swap a reserved node for its equivalent in the opposite direction. This makes it simpler for the train to release nodes while it's travelling after a reverse.

\subsection{Pathfinding}

Pathfinding is done using the original O(v\^2) algorithm by Dijkstra. The edge weight for travelling across the track is given by the physical distance between the different nodes. Reversing is considered another edge, with a configurable penalty. This penalty is very high by default, discouraging reverses unless absolutely necessary. By default, pathfinding ignores track reservations. This was a design decision we made because generally those reservations will change by the time they are reached by the train as the other train moves out of the way.

However, there is a flag that can be passed to the pathfinding algorithm that will assign a very high penalty to reserved edges. This flag is used when a train times out waiting for a reserved piece of track to be freed. In this case, it is likely there is a static train in the way and we will not make progress unless we route around it.

\subsection{Following a path}

Each train has a task dedicated to its control. The train task is responsible for calculating and following paths. It accomplishes this by subscribing to the location stream and performing a set of actions on each position update depending upon its state.

While following a path the train task attempts to reserve upcoming edges until the sum of their distance exceeds the stopping distance of the train plus a safety buffer. If a branch node is successfully reserved, the turnout is switched to the correct direction. Merge nodes are also switched to the correct direction in case we must reverse on top of them. 

If, however, reserving an edge fails, we immediately stop the train. We then enter a waiting state that lasts for three seconds. During these three seconds, we continue to attempt to reserve the track that failed. If we succeed, the train continues on its original path. If we still haven't been able to reserve the track at the end of the three seconds, we attempt to find a different path to our destination.

\subsection{Known Bugs and Limitations}
\begin{itemize}
  \item There are cases where trains estimated positions can cross over each other, even with the reservation system in place.
  \item Sensor attribution sometimes fails when two different trains are expecting to next hit the same sensor. However, this issue is heavily mitigated by the reservation system, which does not allow two trains on the same edge.
\end{itemize}

\end{document}
